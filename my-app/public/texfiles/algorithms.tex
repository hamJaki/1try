\section{Algorithms}
\subsection{Algorithms}
\begin{document}

\section{Algorithms}

Algorithms are precisely defined procedures for solving problems or executing tasks. They are often visualized or written as flowcharts, with a series of steps that include decisions that must be made along the way.

\subsection{Definition}

An Algorithm $A$ is a finite set of well-defined instructions to execute a particular task. Mathematically, it can be expressed as 

\[A = \{a_1, a_2, \dots, a_n\}\]
where $a_i$ is the $i^{th}$ instruction of the algorithm and $n$ is the total number of instructions.

\subsection{Properties of Algorithms}

Many different algorithms can accomplish the same task, but some are much more efficient than others. All algorithms must display the following characteristics:

\begin{enumerate}
\item \textit{Input}: An algorithm has zero or more well-defined inputs.
\item \textit{Output}: An algorithm has one or more well-defined outputs.
\item \textit{Definiteness}: Each instruction defined in an algorithm needs to be clear and unambiguous.
\item \textit{Finiteness}: An algorithm must always terminate after a finite number of steps.
\item \textit{Effectiveness}: An algorithm must be simple enough to be executable.
\end{enumerate}

\subsection{Algorithm Analysis}

In computer science, Algorithm Analysis refers to the process of predicting the computational complexity of an algorithm (how the runtime of the algorithm grows as the input size increases). The 'Big O' notation is typically used to describe the complexity of an algorithm.

\begin{itemize}
\item Constant time complexity, denoted $O(1)$, means that the algorithm performs a fixed number of operations regardless of the input size.
\item Linear time complexity, denoted $O(n)$, means that the algorithm performs a number of operations that scales linearly with the input size.
\item Quadratic time complexity, denoted $O(n^2)$, means that the algorithm performs a number of operations that scales with the square of the input size.
\item Logarithmic time complexity, denoted $O(\log n)$, means that the algorithm performs a number of operations that scales with the logarithm of the input size.
\end{itemize}

In conclusion, algorithms are a fundamental concept embodying the step-by-step procedures needed for a computer to perform some task.

\end{document}

\subsection{The Growth of Functions}
Sure, discussing the topic of functions' growth involves understanding the rate at which a function grows as its input grows, that is, how fast the function's output increases for increasing input.

Let's start with a few basic examples followed by the formal concepts of Big O, Big Omega, and Big Theta notations in LaTeX:

\begin{document}

\section{Growth of Functions}

In mathematics, particularly in calculus, algebra, and analysis, the growth of functions is a topic that is primarily concerned with how fast or how slow a function grows or declines. It forms a significant aspect of complexity analysis in computer science.

Let's consider function $f(x) = x^2$ and $g(x) = x^3$. 

As $x$ grows (as $x \to \infty $), $g(x)$ will eventually outgrow $f(x)$, regardless of how small $x$ is initially. This means $g(x)$ is an upper bound of $f(x)$ or in other words, $g(x)$ grows faster than $f(x)$.

\section{Asymptotic Analysis and Notations}

The most common way to represent the growth of functions, especially in computer science, is using the Big O, Big Omega, and Big Theta notation.

\subsection{Big O Notation}

The Big O notation is used to describe an upper bound on the time complexity of an algorithm, also known as time efficiency. Given two functions $f(x)$ and $g(x)$, we say $f(x)$ is $O(g(x))$ if and only if there exist constants $c > 0$ and $x_0 \geq 0$ such that $|f(x)| \leq c|g(x)|$ for all $x > x_0$.

\subsection{Big Omega Notation}

The Big Omega notation describes a lower bound on the time complexity of an algorithm, also known as space efficiency. We say $f(x)$ is $\Omega(g(x))$, if and only if there exist constants $c > 0$ and $x_0 \geq 0$ such that $|f(x)| \geq c|g(x)|$ for all $x > x_0$.

\subsection{Big Theta Notation}

The Big Theta notation is used when we want to provide both an upper and a lower bound on the time complexity of an algorithm. It provides a tight bound. We say $f(x)$ is $\Theta(g(x))$, if and only if $f(x)$ is $O(g(x))$ and $f(x)$ is $\Omega(g(x))$.

Inside these notations, constants and smaller-ordered terms are usually omitted. For example, if $f(x) = 3x^3 + 2x^2 + 5x + 4$, then $f(x)=O(x^3)$.

\end{document}

There might be cases where a function's growth rate doesn't fit neatly into a Big O, Big Omega, or Big Theta notation. In these cases, complexity is usually expressed in terms of multiple variables or using lesser-known notations like Big-Little-O or Soft-O.


\subsection{Complexity of Algorithms}
Sure, here you are:

```latex
\documentclass{article}
\usepackage[utf8]{inputenc}
\usepackage{amsmath}

\title{Complexity of Algorithms}
\author{Math Expert}
\date{}
\begin{document}

\maketitle

\section{Introduction}
The complexity of an algorithm refers to the amount of resources, such as time and space, that an algorithm requires to solve a problem. In the context of algorithms, complexity is typically considered in terms of time complexity and space complexity.

\section{Time Complexity}
Time complexity of an algorithm quantifies the amount of time taken by an algorithm to run, as a function of the length of the string representing the input. It is commonly estimated by counting the number of elementary steps performed by any run of the algorithm to finish.

The time complexity is generally expressed as $T(n)$, where $n$ is the size of the input. The $O$ notation is used to define an upper bound of an algorithm in the worst-case scenario. For example:

\begin{equation}
T(n) = O(n^2)
\end{equation}

states that the time complexity of the algorithm in the worst-case scenario is proportional to the square of the size of the input.

\section{Space Complexity}
Space complexity of an algorithm quantifies the amount of space or memory taken by an algorithm to run, as a function of the length of the string representing the input. 

The space complexity is generally expressed as $S(n)$, where $n$ is the size of the input. The $O$ notation is used to define an upper bound of an algorithm in the worst-case scenario. For example:

\begin{equation}
S(n) = O(n)
\end{equation}

states that the space complexity of the algorithm in the worst-case scenario is proportional to the size of the input.

It is worth noting that often there is a trade-off between time and space complexity, and a good algorithm efficiently manages this trade-off.

\end{document}
```

This LaTeX code will give you a simple document explaining the complexity of algorithms in terms of both time and space complexities, including the Big O notation. Remember to use a LaTeX compiler to obtain the final output.

