\section{Trees}
\subsection{Introduction to Trees}
Certainly. Here is an introduction to the mathematical concept of trees, written in LaTeX format. In this explanation, it is assumed that the reader already has a basic understanding of set theory:

```latex
\documentclass{article}
\usepackage{amsmath, amssymb, amsfonts}

\begin{document}

\title{Introduction to Trees in Graph Theory}
\author{Math Expert}
\maketitle

\begin{abstract}
This document provides an introduction to the concept of trees in the field of graph theory.
\end{abstract}

A graph $G$ is a pair $(V,E)$ comprising a set $V$ of vertices or nodes together with a set $E$ of edges or arcs. Each edge is a 2-element subset of $V$. 

\section{Trees}

A tree is a special type of graph and is naturally a highly relevant concept in Graph Theory. A \textbf{tree} is a non-empty graph $T = (V, E)$ with the properties:
\begin{itemize}
\item There is a unique simple path between each pair of vertices.
\item The graph contains no simple cycles. % a cycle is simple if the only repeated vertices are the first and last ones
\end{itemize}

Formally, a \textbf{tree} is a graph that is \textit{connected} and has no \textit{cycles}.

\section{Connectedness}

A graph is \textbf{connected} if, given any two vertices, there exists a path between them. Mathematically, we can denote this as: \\ 

\[
\forall u,v \in V,\: \exists \:a\ldots z \in V \:\mathrm{such\:that\:} (u,a),(a,b),\dots,(y,z),(z,v) \in E
\]

\section{Cycles}

A \textbf{cycle} is a closed path, i.e., a path that starts and ends at the same vertex and has at least one edge. A Graph that contains no cycles is an \textit{acyclic} graph.

Therefore, when we have a connected acyclic graph, we have a \textit{tree}. Trees have found numerous applications in computer science, optimization, networking, data organization and several other areas. 

\end{document}
```
 
In this document, we have used the AMS math and symbol packages to represent mathematical entities. The title, author, and abstract sections are chosen for formal presentation and are not compulsory if you are writing an informal note. The items in the 'Trees' and 'Connectedness' sections are formatted as bullet points and mathematical equations, respectively. The 'Cycle' section explains the concept of cycles in graph theory - another important concept for understanding trees.

\subsection{Applications of Trees}
Certainly, here is a basic structure for a LaTeX document covering "Applications of Trees" in Discrete Mathematics:

```tex
\documentclass{article}
\usepackage{amsmath, amssymb}

\begin{document}

\title{Applications of Trees in Discrete Mathematics}
\author{John Doe}
\date{\today}

\maketitle

\begin{abstract}
This document explores the various applications of tree data structures in the field of discrete mathematics. A tree is a non-linear hierarchical structure that represents relationships between various elements or entities.   
\end{abstract}

\section{Introduction}

A tree can be defined in the context of graph theory as an acyclic connection of vertices. In mathematical terms, a tree is an undirected graph $T$ such that for any two vertices $v$ and $w$, there exists exactly one path from $v$ to $w$. 

\begin{equation}
T = \{(v,w) : v,w \in V \text{ and there exists exactly one path from } v \text{ to } w\}
\end{equation}

\section{Applications}

\subsection{Computer Science}
In Computer Science, trees are heavily utilized in data structures and algorithms. For instance, a binary tree data structure can facilitate efficient search, insert, and delete operations.

\subsection{Networks}
Trees are used in network designs, including computer networks, telecommunications, and transportation networks, to find the shortest path and spanning trees.

\subsection{Game Theory}
In game theory, the concept of a game tree is used to mathematically represent a sequential game.

\section{Conclusion}
This document has provided an overview of the wide range of applications of trees in discrete mathematics. The structure and properties of trees make them a versatile and powerful tool in various fields.

\end{document}
```

Please note, this is just a rough outline. You will have to fill in more details for each of the sections and subsections as necessary. You can add appropriate figures, proof, algorithm or pseudocode for the explained topics.


\subsection{Tree Traversal}
Sure, here is an outline for a detailed explanation on tree traversal:

```latex
\documentclass{article}
\usepackage{amsmath,amssymb}
\usepackage{graphicx}
\usepackage[ruled,vlined]{algorithm2e}
\begin{document}

\title{Tree Traversal in Computer Science}
\author{Author}
\maketitle

\textbf{Abstract:} Tree Traversal in computer science refers to various methods of visiting (checking and/or updating) each node in a tree data structure, exactly once. 

\section{Introduction}
There are two major strategies to traversing trees:

\begin{itemize}
    \item Depth-first Search (DFS)
    \item Breadth-first Search (BFS)
\end{itemize}

\section{Depth-first Search}

Depth-first Search (DFS), as the name suggests, is traversal method that starts at the root (or an arbitrary node in a graph, in the case of a graph traversal), and explores as far as possible along each branch before backtracking. This search strategy can be defined using a stack data structure. 

Here is an example algorithm written in pseudo-code:

\begin{verbatim}
DFS (G, v) is
    label v as discovered
    for all directed edges from v to w that are in G.adjacentEdges(v) do
        if vertex w is not labeled as discovered then
            recursively call DFS(G, w)
\end{verbatim}

\section{Breadth-first Search}

Breadth-first search (BFS) is another method for traversing or searching tree or graph data structures. It starts at the tree root (or some arbitrary node of a graph, sometimes referred to as a 'search key') and explores the neighbor nodes at the present depth prior to moving on to nodes at the next depth level.

Below is an implementation of the BFS algorithm given in pseudo-code:

\begin{verbatim}
BFS(G, root) is
    let Q be a queue
    label root as discovered
    Q.enqueue(root)
    while Q is not empty do
        v := Q.dequeue()
        if v is the goal then
            return v
        for all edges from v to w in G.adjacentEdges(v) do
            if w is not labeled as discovered then
                label w as discovered
                Q.enqueue(w)
\end{verbatim}

\section{Applications and Other Types of Traversals}

Tree traversal techniques are integral parts of many computational algorithms in computer science. Such methods can be applied in problems related to searching, routing, connectivity, and even complex operations on expressions, just to name a few. 

There exist other types of traversals too, especially in the field of binary trees, such as in-order, pre-order and post-order traversals, each having its own specific use-cases.

\end{document}
```
In LaTeX, use the `verbatim` or 'algorithm' environment to represent pseudo-codes. Moreover, detailed mathematical expressions can be illustrated using AMS-LaTeX.

\subsection{Spanning Trees}
Sure, "Spanning Trees" is an important topic in Graph Theory. I'll explain this topic in LaTeX format:

\begin{document}

\title{Spanning Trees}
\maketitle

A \textit{spanning tree} of a graph $G$ is a tree that spans $G$ (covers every vertex) and is a subgraph of $G$ (every edge in the tree belongs to $G$). Every connected graph has a spanning tree.

\section{Definition}
A spanning tree $T$ of an undirected graph $G$ is a subgraph that includes all of the vertices of $G$ and is also a tree. In other words, the graph $G$ is connected if and only if there exists a tree that contains all the vertices of G and no additional vertices, i.e., the tree spans the graph. Mathematically, it can be represented as follows:

\[
\text{If } G = (V, E) \text{ and } T = (V', E'), \text{ then } V' = V \text{ and } T \text{ is a tree.}
\]

\section{Properties}
Here are some important properties of spanning trees:

\begin{itemize}
\item A connected graph $G$ with $n$ vertices has a spanning tree with $n-1$ edges.
\item Any connected graph has at least one spanning tree.
\item The total number of spanning trees with $n$ vertices labeled $1, 2, ...,n$ is $n^{(n-2)}$.
\item If we remove an edge from a spanning tree, it will make the graph disconnected.
\item Adding an edge in a spanning tree will form a circuit.
\item Every edge in a graph is a bridge if and only if it belongs to each one of the graph's spanning trees.
\end{itemize}

\section{Applications}
Spanning trees have several applications in computer networks like routing, network design etc. One of the famous algorithms used for creating a minimum spanning tree is \textit{Kruskal’s algorithm} or \textit{Prim's algorithm}.

To create a minimum spanning tree from a graph, as per Kruskal's algorithm, the process is:

\begin{enumerate}
\item Sort all the edges from low weight to high.
\item Keep adding the lowest edge into the spanning tree, if it doesn’t form a cycle.
\item Continue step 2 until all vertices not in the new graph.
\end{enumerate}

This process guarantees the generation of a minimum spanning tree from a connected weighted graph.

\end{document}

Remember to compile this in a LaTeX environment to get a nicely formatted text. LaTeX is widely used in academia for the communication and publication of scientific documents in many fields, including mathematics, statistics, computer science, engineering, chemistry, physics, economics, linguistics, political science, and philosophy.

\subsection{Minimum Spanning Trees}
Sure, let's consider some concepts in graph theory in order to illustrate the concept of Minimum Spanning Trees (MST).

A graph $G = (V, E)$ consists of a set of vertices $V$ and a set of edges $E$. A subgraph $T$ of $G$ is called a spanning tree of $G$ if $T$ is a tree and includes every vertex of $G$. For a connected, undirected graph $G$, a minimum spanning tree is a tree which spans all the vertices and whose sum of weights of all its edges is the least.

Here is how you would specify this in LaTeX format. Quad and newline functions are used to align the equations and provide a neat and readable format:

\begin{verbatim}

\documentclass[preview]{standalone}
\usepackage{amsmath, amssymb}

\begin{document}

A \textit{graph} $G = (V, E)$ consists of a \textit{set of vertices} $V$ and a \textit{set of edges} $E$. 

A subgraph $T$ of $G$ is a \textit{spanning tree} if $T$ is a tree and includes every vertex of $G$, i.e, 

\[ V(T) = V(G) \quad and \quad E(T) \subseteq E(G).\]

For a connected, undirected graph with a real-valued weight function $f: E \rightarrow \mathbb{R}$, a minimum spanning tree is a spanning tree whose sum of weights of all its edges is the least. Formally,

\[ T = \underset{T' \in T(G)}{argmin} \sum_{e \in E(T')} f(e) ,\]

Where $T(G)$ represents all possible spanning trees of $G$.

\end{document}
\end{verbatim}

Algorithms like Prim’s and Kruskal’s are used to find the Minimum Spanning Tree of a graph.

