\section{The Foundations: Logic and Proofs}
\subsection{Propositional Logic}
Sure, here is a detailed explanation with LaTeX formatting:

Propositional Logic is a branch of logic that studies ways of combining or modifying statements or propositions to form more complex statements or propositions. Each proposition has a Boolean value, meaning it is either True (T) or False (F).

To detail these concepts covering the core elements such as basic operations (Conjunction, Disjunction, Negation, Implication, and Equivalence), truth tables, and theorems, we will use LaTeX format.

\begin{itemize}

\item \textbf{Basic Operations}

\begin{itemize}

\item \textbf{Conjunction (AND operator)}: The conjunction of propositions $P$ and $Q$, written as $P \land Q$ is $true$ iff both $P$ and $Q$ are True. Otherwise, it is False.

\item \textbf{Disjunction (OR operator)}: The disjunction of propositions P and Q, denoted as $P \lor Q$ is $true$ if either (or both) of $P$ and $Q$ are true. If both are False, then $P \lor Q$ is False.

\item \textbf{Negation (NOT operator)}: The negation of a proposition $P$, denoted as $\neg P$, is $true$ if $P$ is False, and False if $P$ is True.

\item \textbf{Conditional (Implication operator)}: The implication of propositions $P$ and $Q$, denoted as $P \Rightarrow Q$ is $false$ if $P$ is True and $Q$ is False. In all other cases, it holds True. 

\item \textbf{Biconditional (Equivalence operator)}: The equivalence of propositions $P$ and $Q$, denoted as $P \Leftrightarrow Q$ is $true$ if both $P$ and $Q$ have the same truth value. Otherwise, it is False.

\end{itemize}

\item \textbf{Truth Tables} \\ Truth tables are mathematical tables used in logic to compute the functional values of logical expressions on each of their functional arguments.

Truth Table for NOT:

\begin{tabular}{|c|c|}
\hline
P & $\neg$ P \\
\hline
T & F \\
\hline
F & T \\
\hline
\end{tabular}

Truth table for AND and OR:

\begin{tabular}{|c|c|c|c|}
\hline
P & Q & P $\land$ Q & P $\lor$ Q \\
\hline
T & T & T & T \\
\hline
T & F & F & T \\
\hline
F & T & F & T \\
\hline
F & F & F & F \\
\hline
\end{tabular}

\item \textbf{Theorems}

Some basic theorems in propositional logic include De Morgan’s laws, distributive laws, associative laws, and identity laws. Let $P$, $Q$, and $R$ represent propositional variables. These laws can be written in LaTeX as:

\begin{itemize}
\item \textbf{De Morgan’s laws}: \\
$( \neg ( P \land Q)) \Leftrightarrow ( \neg P \lor \neg Q)$ \\
$( \neg ( P \lor Q)) \Leftrightarrow ( \neg P \land \neg Q)$

\item \textbf{Distributive laws}: \\
$P \land ( Q \lor R) \Leftrightarrow ( P \land Q) \lor ( P \land R)$ \\
$P \lor ( Q \land R) \Leftrightarrow ( P \lor Q) \land ( P \lor R)$

\item \textbf{Identity laws}: \\
$P \land true \Leftrightarrow P$ \\
$P \lor false \Leftrightarrow P$

\end{itemize}

\end{itemize}

These are some of the basic elements of propositional logic explained using LaTeX format.

\subsection{Applications of Propositional Logic}
Sure, here is the explanation in LaTeX format. 

```
\documentclass{article}
\usepackage{amsmath, amssymb}

\begin{document}

\title{Applications of Propositional Logic}
\author{AI Assistant}
\maketitle

Propositional logic is a branch of logic that studies ways of combining and/or changing the truth values of propositions. The applications of propositional logic are seen in various areas of computer science, mathematics, philosophy and linguistics. 

Let us consider one application in Computer Science - namely, in Digital Logic Circuits.

\section*{Digital Logic Circuits}

Digital logic circuits, fundamental to the operation of modern computers, can be described using propositional logic. In these circuits, the values are usually represented by different voltage levels.

For instance, a simple type of circuit is a Boolean circuit where an operation is a logical operation such as AND, OR or NOT. We can represent these operations as functions in propositional logic.

The AND operation, for example, can be represented in propositional logic as a function \(f(p, q)\) which takes the value true (\(1\)) if both \(p\) and \(q\) are true, and false (\(0\)) otherwise. In mathematical terms:

\[
f(p, q) = 
\begin{cases} 
1 & \text{if } p = q = 1, \\
0 & \text{ otherwise.}
\end{cases}
\]

Similarly, we can represent OR and NOT operations. 

\section*{Applications in Mathematics}

Propositional logic also has applications in set theory and algebra, especially when it comes to defining properties of elements in a set. For example, if \(P(x)\) is a property of a set \(X\), we can use propositional logic to make claims such as "For all elements \(x\) in \(X\), \(P(x)\) is true."

\end{document}
```
The given explanation cover two important applications of propositional logic. To apply these concepts in practice, one would need to replace the placeholders such as \(f\), \(p\), \(q\), \(P(x)\), and \(X\) with the actual expressions, predicates, or sets under consideration.





\subsection{Propositional Equivalences}
Propositional logic is the area of mathematics that deals with operations involving logical values. Propositional equivalences involve propositions that have the same truth table, thus are logically equivalent. It's a fundamental tool for proving theorems in mathematics.

\begin{verbatim}

\documentclass{article}
\usepackage{amsmath }

\title{Propositional Equivalences}
\author{Math Expert }
\date{}

\begin{document}

\maketitle

Propositional equivalences involve certain pairs of propositions that have the same truth values in all situations, making them logically equivalent. In other words, the logical content of each proposition in a pair is the same, although the propositions may appear to be different.

Here are some primary propositional equivalences:

1. Identity laws:

\begin{equation}
p \land T \equiv p
\end{equation}

\begin{equation}
p \lor F \equiv p
\end{equation}

2. Domination laws:

\begin{equation}
p \lor T \equiv T
\end{equation}

\begin{equation}
p \land F \equiv F
\end{equation}

3. Idempotent laws:

\begin{equation}
p \lor p \equiv p
\end{equation}

\begin{equation}
p \land p \equiv p
\end{equation}

4. Double negation law:

\begin{equation}
\neg (\neg p) \equiv p
\end{equation}

5. De Morgan's laws:

\begin{equation}
\neg (p \land q) \equiv \neg p \lor \neg q
\end{equation}

\begin{equation}
\neg (p \lor q) \equiv \neg p \land \neg q
\end{equation}

6. Commutative laws:

\begin{equation}
p \land q \equiv q \land p
\end{equation}

\begin{equation}
p \lor q \equiv q \lor p
\end{equation}

These are just several examples of propositional equivalences. The understanding and usage of these laws can significantly simplify the logical expressions, especially in higher areas of mathematics and theoretical computer science.

\end{document}
\end{verbatim}
This generates a document discussing the principle of propositional equivalences along with examples displayed using LaTeX mathematical expressions.

\subsection{Predicates and Quantifiers}
Absolutely! Predicates and Quantifiers are key concepts in mathematical logic. Let's dive into details one by one.

\begin{itemize}
\item \textbf{Predicates:}

A predicate is a statement that contains variables and becomes a proposition when the variables are replaced by particular values. If the domain is \$D\$ and \$P(x)\$ is a predicate over \$D\$, when \$x\$ is in \$D\$, \$P(x)\$ is either true or false.

For example, let \$P(x)\$ denote "x > 3". If \$D = \{1, 2, 3, 4, 5\}\$,
then \$P(2)\$ is false and \$P(4)\$ is true.

\item \textbf{Quantifiers:}

Quantifiers bind the variables in a predicate. The two standard quantifiers are the universal quantifier (\$\forall\$) and the existential quantifier (\$\exists\$).

\begin{itemize}
\item The Universal Quantifier (\$\forall \$): The statement "\$\forall x P(x)\$" is read "for all x P(x)" or "for every x, P(x)", and is true if, and only if, \$P(x)\$ is true for every x in the domain.

For example, if \$D = \{1, 2, 3, 4, 5\}\$ and \$P(x)\$ is "x > 0", then "\$\forall x P(x)\$" is true.

\item The Existential Quantifier (\$\exists \$): The statement "\$\exists x P(x)\$" is read as "there exists an x such that P(x)", and is true if, and only if, \$P(x)\$ is true for at least one x in the domain.

For example, if \$D = \{1, 2, 3, 4, 5\}\$ and \$P(x)\$ is "x > 3", then "\$\exists x P(x)\$" is true.
\end{itemize}
\end{itemize}

These concepts are pillars of mathematical logic and are used extensively in proofs, algorithms, and computational theory.

\subsection{Nested Quantifiers}
In mathematics, we often deal with sets, relations, and functions that involve several variables. In order to express properties about these variables, we employ quantifiers — mathematical symbols that specify the extent to which a predicative function is true over a range of elements. The two most common quantifiers are the universal quantifier (∀) and the existential quantifier (∃).

"Nested Quantifiers" refers to a situation where we have more than one quantifier, and they are arranged in a nested fashion. This means that one quantifier is inside the scope of another one.

The LaTeX format of the universal quantifier is \(\forall\) and the existential quantifier is \(\exists\).

Let's understand the nested quantifiers with some examples.

Example 1: Take the statement: "For all real numbers x, there exists a real number y such that \(y > x\)."

In LaTeX format, it is written as:

\[
\forall x \in \mathbb{R}, \, \exists y \in \mathbb{R} \, \text{such that} \, y > x
\]

Example 2: Another example could be the statement: "For any two different real numbers x and y, there always exists a real number z such that \(x < z < y\)" or \(y < z < x\), depending on whether \(x > y\) or \(y > x\).

In LaTeX, it becomes:

\[
\forall x, y \in \mathbb{R}, \, x \neq y \Rightarrow \exists z \in \mathbb{R} \, \text{such that} \, (x < z < y \, \text{or} \, y < z < x)
\]

The concept of nested quantifiers is essential and widely used in mathematical proofs, especially in subjects like calculus, real analysis and algebra. They describe mathematical facts expressing "for all" or "there exists" type properties that involve more than one variable.

\subsection{Rules of Inference}
Sure, here is your request in LaTeX format:

```
\documentclass{article}
\usepackage{amsmath, amssymb}

\begin{document}

\title{Rules of Inference}
\author{Math Expert}
\date{\today}
\maketitle

\section{Introduction}

The "Rules of Inference" provide the logical foundation for deducing the validity of arguments in mathematics and logic. These rules establish the conditions or operations where, given certain propositions (premises), a conclusion can be reliably drawn.

\section{Common Rules of Inference}

There are several common rules of inference, each representing a valid logical operation:

\subsection{Modus Ponens} If $P$ implies $Q$ and $P$ is asserted to be true, then $Q$ must be true. This is written:

\[
    \begin{array}{ c l }
        P \implies Q & \text{Premise} \\
        P & \text{Premise} \\
        \hline
        Q & \text{Conclusion}
    \end{array}
\]

\subsection{Modus Tollens} If $P$ implies $Q$ and $Q$ is false, then $P$ must also be false. This is written:

\[
    \begin{array}{ c l }
        P \implies Q & \text{Premise} \\
        \neg Q & \text{Premise} \\
        \hline
        \neg P & \text{Conclusion}
    \end{array}
\]

\subsection{Hypothetical Syllogism} If $P$ implies $Q$ and $Q$ implies $R$, then $P$ implies $R$. This is written:

\[
    \begin{array}{ c l }
        P \implies Q & \text{Premise} \\
        Q \implies R & \text{Premise} \\
        \hline
        P \implies R & \text{Conclusion}
    \end{array}
\]

\subsection{Disjunctive Syllogism} If $P$ or $Q$ is true, and $P$ is false, then $Q$ must be true. This is written:

\[
    \begin{array}{ c l }
        P \vee Q & \text{Premise} \\
        \neg P & \text{Premise} \\
        \hline
        Q & \text{Conclusion}
    \end{array}
\]

And there are many more rules of inference. They all follow a similar pattern: Some set of premises leading to a logically certain conclusion.

\end{document}
```

This LaTeX code will present a brief definition of the rules of inference, along with a description of common examples, formatted in proper mathematical symbols.

\subsection{Introduction to Proofs}
Unfortunately, I'm unable to format LaTeX here, but here's an elaboration on "Introduction to Proofs".

(Assume that all mathematical notation, equations and special characters are enclosed within `$$` in a LaTeX document.)

Introduction to proofs is a fundamental topic in mathematics which covers basic techniques for building mathematical arguments to prove the validity of assertions. 

\section{Types of Proofs}

\subsection{Direct Proofs}

Direct proof is the most common technique where the argument flows from the assumptions, through a sequence of logical deductions to the conclusion. For instance, to prove that for all even numbers $n$, $n^2$ is also even.

\subsection{Proof by Contradiction}

Proof by contradiction (reductio ad absurdum) involves supposing that the statement to be proved is false and then deriving a contradiction, thus reinforcing that the statement must be true. For example, the proof that $\sqrt{2}$ is irrational.

\subsection{Proof By Induction}

Proof by induction is a method used to establish that a given statement is true for all natural numbers. It involves two steps:
\begin{itemize}
\item Base Case:  proving that the statement holds for the first natural number.
\item Induction Step:  proving that if the statement holds for some natural number $n$, then the statement also holds for the number $n + 1$.
\end{itemize}

\section{Writing Proofs}

Writing proofs requires a structured argument leading to a conclusion based on some initial assumptions. 

\section{Math Symbols in Proofs}

Proper use of mathematical symbols, notations and abbreviations is necessary in writing mathematical proofs. Some common ones include:
\begin{itemize}
\item $ \forall $: For all
\item $ \exists $: There exists
\item $ \in $: Element of
\item $ \Rightarrow $: Implies
\item $ \iff $: If and only if (iff)
\end{itemize}

\section{Statement and Conclusion}

A mathematical proof always begins with a mathematical statement or proposition, and a conclusion based on the logically ordered sequence of steps.

For instance, to prove the statement “If n is an odd integer, then n^2 is odd”, we assume that n is odd (which means it can be written in the form 2k+1), then showing that $n^2$ = $(2k+1)^2$ = 4k^2 + 4k + 1 = 2(2k^2 + 2k) + 1, which is an odd number. Hence, the statement is proven. The hypotheses is "n is an odd integer" and conclusion is "n^2 is odd".

\subsection{Proof Methods and Strategy}
Creating formal mathematical proofs often relies on a specific methodology or strategy. Here, we discuss several proof methods: direct proof, proof by contradiction, and mathematical induction. Please note that due to limitations in the current system, LaTeX might not be rendered correctly.

1. Direct Proof \- This is one of the most straightforward methods of proving the accuracy of a statement. Direct proofs confirm the legitimacy of a mathematical statement by deducing a conclusion from premised facts, definitions, and accepted generalizations.

e.g., Proving: If \(a\) and \(b\) are both even, then \(a + b\) is even.

\begin{verbatim}
\textbf{Proof:} An even number can be defined as 2k where k is an integer. So, if 
$a = 2m$ and $b = 2n$ for integers $m$ and $n$, then $a + b = 2m + 2n = 2(m+n)$.
Therefore, $a + b$ is equivalent to 2 times the integer $(m+n)$, hence it is even.
\end{verbatim}

2. Proof by Contradiction \- Also known as reductio ad absurdum, this method of proof involves assuming the negation of the statement to be true, and then showing that this leads to an absurd or untenable situation.

e.g., Proving: sqrt(2) is an irrational number.

\begin{verbatim}
\textbf{Proof:} Assume for contradiction that $\sqrt{2}$ is rational means it can be 
expressed as $p/q$ where $p$ and $q$ are coprime numbers (with no common divisors 
except 1). So, $\sqrt{2} = p/q  \Rightarrow 2 = (p^2)/(q^2) \Rightarrow p^2 = 2q^2$.
Since 2 divides $p^2$, 2 divides $p$. Let $p = 2r$, then $(2r)^2 = 2q^2 \Rightarrow 
4r^2 = 2q^2 \Rightarrow q^2 = 2r^2$. This implies 2 divides $q$, contradicting 
the assumption that $p$ and $q$ are coprime. Therefore, $\sqrt{2}$ is irrational.
\end{verbatim}

3. Mathematical Induction \- Induction is quite unique as it allows one to have a finite framework to establish an infinite set of statements. It is typically used to prove statements about all numbers and is often used in iterative functions where each step is based on the previous one.

e.g., Proving: 1 + 2 + 3 + ... + n = n(n+1)/2 for all positive integers n.

\begin{verbatim}
\textbf{Proof:} 
    \begin{itemize}
        \item \textbf{Base Step:} For $n=1$, $1 = 1(1+1)/2$ is true.
        \item \textbf{Inductive Step:} Assume it holds for $n=k$, i.e., 
        $1 + 2 + ... + k = k(k+1)/2$.
        Now, consider $n = k + 1$: 
        $1 + 2 + ... + k + (k+1) = [k(k+1)/2] + (k+1) = (k+1)(k/2 + 1) =
        (k+1)(k+2)/2$.
    \end{itemize}
So, by the principle of mathematical induction, the formula holds for all positive integers $n$.
\end{verbatim}
The logic and strategies behind these methods are vital when solving more complex problems. They don't always work directly and sometimes need to be combined, but understanding them can help in crafting effective proofs.

